\documentclass[a4paper, 10pt]{article}
\usepackage[utf8x]{inputenc}
\usepackage{graphicx}
\usepackage{geometry}
\usepackage{amsmath}
\usepackage{mathenv}
\usepackage{amssymb}
\usepackage{amsfonts}
\usepackage{mathrsfs}
\usepackage{textcomp}
\geometry{hmargin = 2.5cm, vmargin = 1.5cm}

% OPENING
\title{SY09 - TP04\\Analyses discriminantes quadratique et linéaire}
\author{Bertrand Bon - Antoine Hars}

\begin{document}

\maketitle

\section*{Introduction}


\section*{Exercice 1 : Règle de Bayes.}

\subsection*{}

\subsection*{}

\section*{Exercice 2 : Analyse discriminante sur les données \textit{Crabes}.}

\subsection*{1. Expliquer en deux lignes ce que fait chacune des fonctions :}
\textbf{lda :} Cette fonction sert à effectuer l'analyse discriminante linéaire qui cherche à détecter sur la matrice de covariance est singulière.\\
Cette fonction peut être appelée avec en paramètre une formule, un data frame, une matrice.\\
\textbf{qda :} Cette fonction est utilisée pour effectuer l'analyse discriminante quadratique en utilisant une décomposition QR qui retournera un message d'erreur si la variance du groupe est singulière pour chaque groupe.\\
\textbf{contour :} Cette fonction est utile pour créer un graphe de contour ou pour ajouter une ligne de contour à un graphe existante.\\
\textbf{sample :} Cette fonction nous permet de récupérer un échantillon de taille spécifiée d'éléments de l'ensemble X.\\
\textbf{predict :} Predict() est une fonction générique de prédictions de modèle.\\
\textbf{predict.lda :} Cette fonction classifie des observations multi-variables en utilisant l'analyse discrimante linéraire.\\
\textbt{Comparaison entre predict et predict.lda :} predict est générique et predict.lda non.\\ \\
\subsection*{}
\subsection*{}
\subsection*{}

\section*{Exercice 3 : }

\subsection*{}
\subsection*{}
\subsection*{}

\section*{Conclusion}

\end{document}
